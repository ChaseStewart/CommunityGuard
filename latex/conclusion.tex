\section{Conclusions and Future Work}
\label{sec:concl}

\subsection{Conclusion}
\label{sec:concl:main}
As network security will only become more crucial over time, we aspire to a future where users' personal area networks and devices defend themselves and each other from emerging threats. To this end, we introduce \sysname, an in-line intrusion prevention system where a \nodename in each subnet shares new threats with nodes in other subnets. This provides a sort of herd immunity against new attacks. As soon as an attack claims a victim, all of the victim's peers will be informed, and will repel the same type of attack against their own subnets. We introduce a high-level architecture for the entire system, as well as design considerations for individual devices and for the \servname. We consider some potential attack vectors against the server and show how the design of the server mitigates such attacks.  \\
To demonstrate the novelty and benefits of \sysname, we implemented a very small scale version of the architecture. We used two BeagleBone Black embedded systems to each host a \nodename and created an \sysname using a medium-class Amazon Web Services EC2 instance. Using this minimum-viable model, we configured and implemented the necessary software on each device, and then used this setup to test the architecture and the performance of the \sysname system. Through this implementation, we were able to do the following:

\begin{itemize}
    \item Create the \servname server tables and input processing.
    \item Describe and address several potential attack models against the \servname server design through its architecture and security permissions.
    \item Demonstrate alerting and blocking traffic using the \nodename, as well as push to/pull from the \servname server.
    \item Demonstrate that using the prototype \nodename with sub-optimal hardware had a low impact on a user's internet browsing performance, and that there is much room for improvement in this area.
\end{itemize}

\subsection{Further Work}
\label{sec:concl:further}
We have tested this system on a very small scale utilizing two BeagleBone Black devices communicating to one AWS server- we can think of several areas of further development for \sysname, including the following:

\begin{enumerate}
\item Creating a very large-scale implementation ( \textgreater 10,000 \nodenames) and testing its performance and security
\item Breaking up \servnames into regions and/or caching the blacklist so as to bound its growth
\item Incorporating \sysname into a router rather than an additional device
\item Re-implementing and performance-testing the \nodename with more specialized hardware
\item Optimizing the Snort configuration and rules to fit the right hardware and threat model
\item Running multiple Snort instances in a multi-core device to prevent downtime while Snort rules reload
\end{enumerate}

%Some such area There remains design work to be done on the scaling of the project as we believe that the benefit of this system scales proportionally to the proportion of subnets utilizing \sysname. It is still left to be determined whether there would be only one \servname server or perhaps several \servname servers separated by region, and it is unknown how such a design would affect the performance and security of the \sysname system. \\

%Another great follow-up to this project would be to re-implement the project with a more specialized hardware, or perhaps even to investigate running a \nodename as code on a router itself. We believe that the performance issues that result from running an in-line \nodename would be alleviated with the introduction of a more specialized hardware device. \\

%As the number of \nodenames and the number of reported threats would rise over time, the list of threats within the \servname might become quite unmanageable. To solve this problem, we imagine a sort of caching on threats based on frequency and geographical prevalence of attacks. Additionally, a few levels of threat caching would improve performance while still keeping the \nodename current and robust to attacks. \\

%Finally- the set of rules used by Snort are not run-time configurable as it stands, but there are ways to achieve this such as running two instances of Snort simultaneously with shared memory and a multi-core processor. A future implementation with specialized software will solve this issue and keep the \nodename's IPS running at all times.