\section{Conclusions and Future Work}
\label{sec:concl}

\subsection{Conclusion}
\label{sec:concl:main}
As network security will only become more crucial over time, we aspire to a future where users' personal area networks and devices defend themselves and each other from emerging threats. To this end, we introduce Avant-Guard, an in-line intrusion prevention system where a guardian node in each subnet shares new threats with nodes in other subnets. This provides a sort of herd immunity against new attacks. As soon as an attack claims a victim, all of the victim's peers will be informed, and will repel the same type of attack against their own subnets. We introduce a high-level architecture for the entire system, as well as design considerations for individual devices and for the Avant-Guard server. We consider some potential attack vectors against the server and show how the design of the server mitigates such attacks.  \\
To demonstrate the novelty and benefits of the project, we implemented a very small scale version of the architecture. We used two BeagleBone Black embedded systems to each host a guardian node and created an Avant-Guard Server using a medium-class Amazon Web Services EC2 instance. Using this minimum-viable model, we configured and implemented the necessary software on each device, and then used this setup to test the architecture and the performance of the Avant-Guard system. Through this implementation, we were able to do the following:

\begin{itemize}
    \item Create the AGS server tables and input processing.
    \item Describe and address several potential attack models against the AGS server design through its architecture and security permissions.
    \item Demonstrate alerting and blocking traffic using the Guardian Node, as well as push to/pull from the AGS server.
    \item Demonstrate that even using our the prototype device with sub-optimal hardware had very minimum impact on performance.
\end{itemize}

\subsection{Further Work}
\label{sec:concl:further}
We have tested this project on a very small scale utilizing two BeagleBone Black devices communicating to one AWS server. There remains design work to be done on the scaling of the project as we believe that the benefit of this system scales proportionally to the proportion of subnets utilizing Avant-Guard. It is still left to be determined whether there would be only one Avant-Guard server or perhaps several Avant Guard servers separated by region, and it is unknown how such a design would affect the performance and security of the Avant-Guard system. \\

Another great follow-up to this project would be to re-implement the project with a more specialized hardware, or perhaps even to investigate running a guardian node on a router itself. Time constraints have kept us from delivering more than a proof-of-concept, but as mentioned above, we believe that the performance issues that result from running an in-line guardian node would be alleviated with the introduction of a more specialized hardware device. \\

As the number of guardian nodes and the number of reported threats would rise over time, the list of threats within the AGS might become quite unmanageable. To solve this problem, we imagine a sort of caching on threats based on frequency and geographical prevalence of attacks. This concept would take some further development, but perhaps a few levels of threat caching would improve performance while still keeping the guardian node current and robust to attacks. \\

Finally- the set of rules used by Snort are not runtime configurable as it stands, but there are ways to achieve this such as running two instances of Snort simultaneously with shared memory and a multi-core processor. A future implementation with specialized software will solve this issue and keep the Guardian node's IPS running 100\% of the time.