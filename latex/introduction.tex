\section{Introduction}
\label{sec:intro}

As the Internet continues to control and define more aspects of the physical world, the consequences of Internet attacks also continue to scale. Network attacks, once an annoyance or hassle, can now translate into a loss of money, a source of physical harm, or even widespread chaos. In the past few years alone, we have seen network attacks destroy nuclear reactor development \cite{stux}, hold hospitals ransom\cite{ransom}, and even control moving vehicles\cite{carhack}. A particularly weak area increasingly targeted by attackers is the Internet of Things. It is hard to imagine the influence of computer networks and the prevalence of Internet ever waning. Therefore, creating permanently secure and stable networks is another technical problem that must be solved before the Internet of Things firmly takes hold. \\

Perhaps a good source of inspiration for the problem of securing these innumerable, complex networks is the physical layer of the Internet itself. Despite the daunting physical, link, and protocol issues of running cables throughout the countries of the world, it is amazing that modern users do not even need to know how the Internet works to benefit from its many services. It is very encouraging to imagine that one day network security could be so well co-ordinated. This workshop project, known as Avant-Guard, begins to address the prospects of a self-policing network through relying on the collaborative aspects of the net in order to catch ever-more complicated malware and threats, and to block malicious traffic from ever reaching its intended destinations. \\

Of the attacks mentioned above, one stands out as especially poised to grow in damage and scale. Distributed Denial of Service, or DDoS, is a common network attack in which a user attempts to make an extremely high number of resource requests in a short time in order to prevent others from accessing the resource. The strength of a DDoS attack is proportional to the number of resources an attack can leverage against its opponent and the time for which they can sustain their resource requests. For this reason, the Internet of Things looming on the horizon (which promises as much as 400 zettabytes of data by 2018 \cite{zetta}) has already given attackers a huge advantage in sustaining record-breaking DNS attacks against increasingly worrying targets. In events such as the attack on Dyn \cite{dyn}, a moderate amount of traffic passed from an unprecedented number of unique devices (mainly IOT cameras) to create an overwhelming force. One idea for defeating these potentially-enormous DDoS attacks will be to stop the traffic at the weaker end of its path; the source subnet. Avant-Guard demonstrates the concept of dropping DDoS traffic at source subnets as a possible solution to this daunting problem.