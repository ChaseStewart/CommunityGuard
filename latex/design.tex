\section{System Design}

% TODO I know I need to crop this
\begin{figure}
    \centering
    \includegraphics[width=0.95\linewidth]{figs/macro.png}
    \caption{Avant-Guard macro-level architecture}
    \label{fig:macro}
\end{figure}


\label{sec:design}

Figure~\ref{fig:macro} shows the high-level design of Avant-Guard. The keystone of the proposed architecture is the Avant-Guard IPS device - also known as the Guardian node, which had to be placed in a position where it could view the entire network traffic flowing into and out of a user's network, and block all suspicious traffic. The addition of the device had to be as non-intrusive as possible, requiring no modification in Modem or Router configuration. \\

Based on these factors, the ideal placement of the Guardian node was chosen between the Modem and the Router. For networks using a Modem and a Router on the same device, we foresee that a production scale implementation of Avant-Guard would place the Guardian hardware within the same box as the Modem and the Router, therefore maintaining the same architecture as shown in Figure~\ref{fig:macro}. Such Guardian nodes placed at the entry point of home networks can create a solid net of protection that communicates threats as soon as they emerge. It must be noted that the responsiveness and accuracy of Avant-Guard in deterring threats, and especially its ability to mitigate DDoS attacks, is directly proportional to the number of subnets that use this system; ideally there would be a Guardian node within every household and business subnet.

\subsection{The Guardian Node}
\label{sec:design:guardian}

\subsubsection{Hardware Design}
The BeagleBone Black, a reasonably powerful embedded device, was chosen as the Guardian node since it satisfied the minimum requirements to implement the envisioned end product. The on-board 10/100 Mbps Ethernet port provided one network interface, while another interface was added using an USB to 10/100 Mbps Ethernet adapter. Obviously, in the absence of time and budget restrictions, the best case implementation would have had two on board GigE ports- however this prototype is sufficient to display the core mechanics of the Guardian Node. 

\subsubsection{Software Design}
\label{sec:design:software}

The software design of a Guardian node is depicted in Figure~\ref{fig:guardian}. The Guardian has a DHCP server configured so that a Router can be added behind it. It passes all traffic from one network interface to the other. The Guardian node runs Snort \cite{Roesch:1999:SLI:1039834.1039864}, a powerful Intrusion Detection/Prevent tool, to monitor traffic at both interfaces. A set of Snort rules are configured to block any suspicious traffic that satisfies any of the rules. Snort can perform protocol analysis, content searching/matching, and can be used to detect a variety of attacks and probes, such as buffer overflow, stealth port scans, CGI attacks, SMB probes, OS fingerprinting attempts, and many more. It also provides a black-list and white-list functionality that can be used to conditionally block and unblock malicious IP addresses. Snort was chosen as the IPS due to its large amount of documentation, open source software and up-to-date rule-lists, and its high regard in security communities. \\

Communication of malware alert between various Guardian nodes is a key aspect of Avant-Guard. This communication is done on a periodic basis by a set of three cron jobs running on the Guardian node. These cron jobs do the following:

\begin{itemize}
    \item Keep Snort's general rules up-to-date by pulling rules and IPs from rule repositories.
    \item Push this particular guardian node's suspicious traffic to the AGS and also to pull new rules and blocked IPs from the AGS.
    \item Check for DDoS server beacons and generate new anti-DDoS rules if necessary.
\end{itemize}

The first cron job runs periodically and reads a log created by Snort that contains IP address of sources that have sent bad traffic. It then parses the source IP addresses present in the log, and updates the same to the database on the Avant-Guard Server (described in the following section). \\

% TODO I know I need to crop this
\begin{figure}
    \centering
    \includegraphics[width=0.95\linewidth]{figs/ddostable.png}
    \caption{Outbound DDoS prevention Algorithm}
    \label{fig:ddostable}
\end{figure}

% TODO I know I need to crop this
\begin{figure}
    \centering
    \includegraphics[width=5cm, height=6cm]{figs/guardiannode.png}
    \caption{Anatomy of a guardian node}
    \label{fig:guardian}
\end{figure}

A second cron job runs periodically to fetch information from a DDoS watch list available on the Avant-Guard Server as 
part of the Outgoing DDoS prevention mechanism (server-side is described in Section \ref{sec:design:serverddos}). If the database provides new IP addresses (for which rules do not exist on the Guardian yet) that are currently being DDoSed, the cron job creates and adds Snort rules to drop all such outgoing DDoS traffic, thus clipping off the attack at the compromised source itself. 
This algorithm is depicted in Figure ~\ref{fig:ddostable}; Such a mechanism, when repeated over all Guardian nodes, can prevent a DDoS attack such as the DYN DDoS attack. \\

Finally, the third cron job runs once every day at some random time and fetches the latest bad IP list from the Avant-Guard Server and adds it to the Snort IP Blacklist. It also fetches Snort rules from Emerging Threats ~\cite{emerging} to defend against new and emerging attack patterns. These three cron jobs combined with Snort running on all Guardian nodes establish a system of protection that learns, communicates and has the capacity to prevent most attacks.

\subsection{Avant-Guard Server}
\label{sec:design:server}

% TODO I know I need to crop this
\begin{figure}
    \centering
    \includegraphics[width=0.95\linewidth]{figs/blacklistserver.png}
    \caption{AGS push and pull architecture}
    \label{fig:blacklistserver}
\end{figure}

The Avant-Guard server is the source of intelligence for the Guardian nodes that it services. The server runs a database and a couple of cron job timed events to process the information within its database. This server uses its database tables to aggregate threats presented by all collected guardian nodes, process them to determine valid threats, and then present them for individual guardian nodes to read and process. As this server is an obvious line of attack for one who would wish to halt the Avant-Guard service, it is foreseen that the server will be elastically-scaling and secure in a production level implementation. 

\subsubsection{Attack Models}
\label{sec:design:attacks}
We assume that a user will not attempt to attack or compromise their own Guardian node since our system aims to protect users who are unaware that their devices have been compromised. Furthermore, the Guardian node itself can be fortified through a hardened OS image and security best practices. This leaves a malicious user two possible vectors of attack for causing harm to a particular user- either attempting to get harmful traffic through the Guardian node, or else attacking the AGS itself. 
We designed the functionality of the AGS while keeping the following attack models in mind:

\begin{enumerate}
\item A malicious user may attempt to add useful IPs to the blacklist to block them for users
\item A malicious user may attempt to remove blacklisted IPs from the blacklist
\item A malicious user may attempt to read user data from the list
\end{enumerate}

\subsubsection{Server Blacklist architecture}
\label{sec:design:blacklist}
When any guardian node is created, a MD5 hash of the node's MAC address and a random nonce stored on the Guardian node's memory is entered into the mac_addr_registry table, as shown in Figure ~\ref{fig:blacklistserver}. This hash value will serve as a key for writes, to ensure that writes can only be initiated from a valid Guardian node.

The guardian node is only allowed write permission to the IPv4_input table and read permission from the IPv4_output table and DDoS_output table. This policy prevents attack models 1, 2, and 3 and ensures that even if a malicious user were to reverse-engineer the device to discover a means to contact the AGS directly, the amount of damage they could cause is minor. When a guardian node is scheduled to push suspicious traffic to the AGS, it is allowed to write only into the IPv4_input table. It will write the following pair to the input table: \[\{ip_{suspicious}, MD5(int(MAC addr), nonce)\}\] 

The AGS server will process the input table to recalculate new threats at a certain frequency, and then drop the current input table. For each entry in the input table, the server first checks the MD5 passed by the guardian node to ensure that this key is already within the mac_addr_registry. The entries that do not have a valid MD5 hash are dropped, leaving only the valid requests. For all valid requests, a second MD5 hash is computed, this time a hash of both the previous MD5 result and now also the IP address declared as suspicious \[MD5(\{ip_{suspicious}, MD5(int(MAC addr), nonce)\}) \]. The AGS checks whether this hash is already in the IPv4_master_list server, which is a list of the MD5, IP combinations that have previously been entered- if this is a new suspicious IP, or it is a known suspicious IP being reported by a new and valid Guardian node, it will be pushed to the IPv4_output table. This second hash function ensures that a given guardian node receives only one vote towards a given IP address, as a means to prevent against attack model 1. 

If this IP is a new entry, it will be entered into the table with a count of 1; if it already exists within the table, its count will be incremented. Once the count tallied against the IP reaches a sufficiently high number proportional to the number of reported threats, the IP will be pushed as output to requesting Guardian nodes, and blocked automatically by all Guardians. Relying on a count before pushing a suspicious IP also helps prevent attack model 1 from occurring. 

\subsubsection{Server Outgoing DDoS prevention}
\label{sec:design:serverddos}

As mentioned above in Section ~\ref{sec:design:software}, the AGS also provides a method to prevent outgoing DDoS traffic from a Guardian node's subnet out to a DDoS target. This system is much more simple than the blacklist method described above: in this case, the DDoS victim list will be edited by Avant-Guard administrators, and only reading will be allowed for all Guardian nodes. Avant-Guard administrators will confirm the existence of a DDoS attack through an alternate channel, possibly a third party DDoS \cite{DDoSPreventionTools} working with Avant-Guard administrators, and then add the relevant IPs or CIDR ranges to the IPv4_ddos table. Guardian nodes will use the cron described above to add and remove rules to prevent outgoing DDoS attacks.